

%\documentclass[11pt, oneside]{article}   	
%\usepackage{geometry}    
%\geometry{letterpaper}                 		
\input preamble.tex
\newcommand{\ig}[2][width=4in]{\includegraphics[#1]{#2}}    		
\usepackage{graphicx}					
\usepackage{amssymb}
\usepackage{pgfplotstable}
\usepackage{float}
\begin{document}

\header {\today}							
\title{Relativistic Electron Momentum}
\author{Ekta Patel \& Brandon Booth-Dunbar}



\section{Abstract}
\begin{em}
%Brandon
\end{em}

\section{Intro}
%Brandon

\section{Theory}
%Ekta
In Newtonian mechanics, kinetic energy and momentum are given by the following equations of motion:
\begin{equation}KE= \frac{1}{2}mv^2 \end{equation}
\begin{equation} p=mv \end{equation}
\begin {equation}KE=\frac{p^2}{2m} \end{equation}

However, when dealing with particles that are moving at the speed of light, relativistic motion must be considered. When the velocity of a particle,v, approaches the speed of light, c, the equations for energy and momentum become:
\begin{equation} E=\gamma mc^2\end{equation}
\begin{equation} p=\gamma mv\end{equation}
where $\gamma$ is:
\begin{equation} \gamma= \frac{1}{\sqrt{1-(\frac{v}{c})^2}}\end{equation}
Therefore, total relativestic energy of a particle can be written as:
\begin{equation} E=m_0c^2+KE=\gamma mc^2\end{equation}
The first term of Equation 7 represents the rest-mass energy of the particle, which is an electron for the purpose of this lab. Relativistic energy is also commonly expressed in terms of momentum as given by Equation 8:

\begin{equation}E^2=p^2c^2+m^2c^4\end{equation}
%talk about how bismuth changes to lead, then the binding energy. then show the following calculations of KE and then B. also talk about using the classical form of momentum due  to the equivalent force of the magnetic field 
We know KE from the lab manual in units of MeV, so we can solve for p:
\begin{equation}KE+mc^2=\sqrt{p^2c^2+m^2c^4}\end{equation}
\begin{equation}(KE+mc^2)^2=p^2c^2+m^2c^4\end{equation}
\begin{equation}(KE+mc^2)^2-m^2c^4=p^2c^2\end{equation}
\begin{equation}\frac{(KE+mc^2)^2-m^2c^4}{c^2}=p^2\end{equation}
\begin{equation}\frac{p=\sqrt{(KE+mc^2)^2-m^2c^4}}{c}\end{equation}
We are given:
\begin{equation}p=qRB\end{equation}
Where R is the radius of the detector-source apparatus, which we have measured to be .0290m:
\begin{equation}\frac{p=\sqrt{(KE+mc^2)^2-m^2c^4}}{c}=eB(.0290m)\end{equation}
Solving for B, we can estimate our magnetic fields needed to see the electron pulses:
\begin{equation}B=\frac{\sqrt{(KE+mc^2)^2-m^2c^4}}{ce(.0290m)}\end{equation}
Before solving for B, we must account for the energy lost by the electron due to k-shell binding energy which is $\sim$88keV:
\begin{equation}KE=1.064MeV-0.088005MeV=0.975995\end{equation}
Now, we can substitute all of our values into equation 11 to obtain:
\begin{equation} B=0.1606T=1.606kG\end{equation}
For the second electron pulse with .5689MeV:
\begin{equation}KE=0.5689MeV-.0.088005MeV=MeV\end{equation}
Once again, using equation 11:
\begin{equation}B=0.0979T=0.979kG\end{equation}

Remember to consider the fact that the electron loses energy as it travels in the air when doing error analysis!

\section{Experimental Methods}
%Brandon
\subsection{Apparatus}

\subsection{Calibration}
The calibration of the apparatus is the most detail sensitive step in the experiment.  While calibrating the detector system if you set the amplifier too high or the single channel analyzer too low you will get very large background counts from the sensor and willnot reliably be able to detect the peaks of electron emission from the �Byzanthium? source.  On the other hand if the discriminator is set too high or the amplifier to low you will not get enough counts to clearly define a peak when you perform your measurements. 
\subsection{Procedure}

\section{Results $\&$ Discussion}
\begin{figure}[H]
\begin{center}
\includegraphics[width=6 in]{figure3.pdf}
\caption{Average of first three runs of data, including error on the number of counts and background noise}
\end{center}
\end{figure}
\begin{figure}[H]
\begin{center}
\includegraphics[width=6 in]{figure4.pdf}
\caption{Run 4 of data with extra observations near the theoretical peaks in magnetic field including background noise.}
\end{center}
\end{figure}
\section{Conclusion}
%Ekta & Brandon
\section{Appendix}
Data tables and graphs.
\begin{table}[h!]
\caption{Run 1}
\begin{tabular}{|c|c|c|} \hline
Field	(kG)&	Counts	&	Error	\\	\hline
0.6009	&	10	&	3.16	\\	\hline
0.6500	&	26	&	5.10	\\	\hline
0.7009	&	20	&	4.47	\\	\hline
0.7505	&	25	&	5.00	\\	\hline
0.8008	&	32	&	5.66	\\	\hline
0.8507	&	28	&	5.29	\\	\hline
0.9006	&	35	&	5.92	\\	\hline
0.9503	&	44	&	6.63	\\	\hline
1.0002	&	36	&	6.00	\\	\hline
1.0525	&	51	&	7.14	\\	\hline
1.1067	&	40	&	6.32	\\	\hline
1.1498	&	18	&	4.24	\\	\hline
1.2000	&	27	&	5.20	\\	\hline
1.2546	&	31	&	5.57	\\	\hline
1.3061	&	29	&	5.39	\\	\hline
1.3528	&	22	&	4.69	\\	\hline
1.4012	&	20	&	4.47	\\	\hline
1.4503	&	25	&	5.00	\\	\hline
1.5016	&	41	&	6.40	\\	\hline
1.5526	&	50	&	7.07	\\	\hline
1.6018	&	79	&	8.89	\\	\hline
1.6512	&	101	&	10.05	\\	\hline
1.7022	&	74	&	8.60	\\	\hline
1.7554	&	61	&	7.81	\\	\hline
1.8021	&	40	&	6.32	\\	\hline
1.8524	&	24	&	4.90	\\	\hline
1.9022	&	20	&	4.47	\\	\hline
1.9495	&	27	&	5.20	\\	\hline
2.0010	&	15	&	3.87	\\	\hline
\end{tabular}
\end{table}
\begin{table}[h!]
\caption{Run 2}
\begin{tabular}{|c|c|c|} \hline
Field	&	Counts	&	Error	\\ \hline
0.6009	&	14	&	3.74	\\ \hline
0.6500	&	17	&	4.12	\\ \hline
0.7004	&	21	&	4.58	\\ \hline
0.7520	&	21	&	4.58	\\ \hline
0.8014	&	33	&	5.74	\\ \hline
0.8502	&	25	&	5.00	\\ \hline
0.9025	&	37	&	6.08	\\ \hline
0.9505	&	36	&	6.00	\\ \hline
1.0002	&	41	&	6.40	\\ \hline
1.0525	&	38	&	6.16	\\ \hline
1.1004	&	33	&	5.74	\\ \hline
1.1501	&	33	&	5.74	\\ \hline
1.2017	&	29	&	5.39	\\ \hline
1.2506	&	21	&	4.58	\\ \hline
1.3006	&	26	&	5.10	\\ \hline
1.3509	&	35	&	5.92	\\ \hline
1.4012	&	19	&	4.36	\\ \hline
1.4506	&	30	&	5.48	\\ \hline
1.5030	&	39	&	6.24	\\ \hline
1.5497	&	60	&	7.75	\\ \hline
1.6015	&	84	&	9.17	\\ \hline
1.6531	&	92	&	9.59	\\ \hline
1.7008	&	80	&	8.94	\\ \hline
1.7535	&	66	&	8.12	\\ \hline
1.8009	&	58	&	7.62	\\ \hline
1.8501	&	36	&	6.00	\\ \hline
1.9009	&	29	&	5.39	\\ \hline
1.9510	&	22	&	4.69	\\ \hline
2.0019	&	16	&	4.00	\\ \hline
\end{tabular}
\end{table}
\begin{table}[h!]
\caption{Run 3}
\begin{tabular}{|c|c|c|} \hline
Field	&	Counts	&	Error	\\ \hline
0.6006	&	19	&	4.36	\\ \hline
0.6496	&	20	&	4.47	\\ \hline
0.7028	&	27	&	5.20	\\ \hline
0.7513	&	26	&	5.10	\\ \hline
0.8015	&	17	&	4.12	\\ \hline
0.8508	&	22	&	4.69	\\ \hline
0.9008	&	30	&	5.48	\\ \hline
0.9509	&	53	&	7.28	\\ \hline
1.0015	&	31	&	5.57	\\ \hline
1.0498	&	36	&	6.00	\\ \hline
1.1000	&	42	&	6.48	\\ \hline
1.1508	&	30	&	5.48	\\ \hline
1.2014	&	24	&	4.90	\\ \hline
1.2508	&	35	&	5.92	\\ \hline
1.3047	&	16	&	4.00	\\ \hline
1.3518	&	29	&	5.39	\\ \hline
1.4011	&	22	&	4.69	\\ \hline
1.4515	&	33	&	5.74	\\ \hline
1.5027	&	41	&	6.40	\\ \hline
1.5505	&	67	&	8.19	\\ \hline
1.6025	&	88	&	9.38	\\ \hline
1.6518	&	92	&	9.59	\\ \hline
1.7015	&	81	&	9.00	\\ \hline
1.7518	&	50	&	7.07	\\ \hline
1.8043	&	40	&	6.32	\\ \hline
1.8519	&	14	&	3.74	\\ \hline
1.9020	&	23	&	4.80	\\ \hline
1.9514	&	21	&	4.58	\\ \hline
2.0021	&	14	&	3.74	\\ \hline
\end{tabular}
\end{table}
\begin{table}[h!]
\caption{Run 4}
\begin{tabular}{|c|c|c|} \hline
Field (kG)	&	Counts	&	Error	\\ \hline
2.0024	&	25	&	5.00	\\ \hline
1.9532	&	25	&	5.00	\\ \hline
1.9019	&	25	&	5.00	\\ \hline
1.8519	&	34	&	5.83	\\ \hline
1.8253	&	29	&	5.39	\\ \hline
1.8006	&	40	&	6.32	\\ \hline
1.7750	&	58	&	7.62	\\ \hline
1.7503	&	55	&	7.42	\\ \hline
1.7251	&	74	&	8.60	\\ \hline
1.7009	&	94	&	9.70	\\ \hline
1.6902	&	92	&	9.59	\\ \hline
1.6801	&	93	&	9.64	\\ \hline
1.6701	&	105	&	10.25	\\ \hline
1.6603	&	122	&	11.05	\\ \hline
1.6505	&	100	&	10.00	\\ \hline
1.6001	&	92	&	9.59	\\ \hline
1.5903	&	83	&	9.11	\\ \hline
1.5805	&	92	&	9.59	\\ \hline
1.5705	&	69	&	8.31	\\ \hline
1.5602	&	75	&	8.66	\\ \hline
1.5517	&	72	&	8.49	\\ \hline
1.5254	&	45	&	6.71	\\ \hline
1.5038	&	33	&	5.74	\\ \hline
1.4759	&	29	&	5.39	\\ \hline
1.4507	&	27	&	5.20	\\ \hline
1.4249	&	35	&	5.92	\\ \hline
1.4019	&	33	&	5.74	\\ \hline
1.3514	&	20	&	4.47	\\ \hline
1.3035	&	36	&	6.00	\\ \hline
1.2501	&	26	&	5.10	\\ \hline
1.2041	&	28	&	5.29	\\ \hline
1.1528	&	20	&	4.47	\\ \hline
1.1255	&	39	&	6.24	\\ \hline
1.1027	&	32	&	5.66	\\ \hline
1.0896	&	34	&	5.83	\\ \hline
1.0805	&	43	&	6.56	\\ \hline
1.0702	&	47	&	6.86	\\ \hline
1.0606	&	42	&	6.48	\\ \hline
1.0509	&	41	&	6.40	\\ \hline
1.0246	&	43	&	6.56	\\ \hline
1.0011	&	39	&	6.24	\\ \hline
0.9531	&	39	&	6.24	\\ \hline
0.9050	&	43	&	6.56	\\ \hline
0.8511	&	30	&	5.48	\\ \hline
0.8022	&	40	&	6.32	\\ \hline
0.7521	&	21	&	4.58	\\ \hline
0.7020	&	28	&	5.29	\\ \hline
0.6498	&	20	&	4.47	\\ \hline
0.6004	&	26	&	5.10	\\ \hline
\end{tabular}
\end{table}
\begin{figure}[h!]
\begin{center}
\includegraphics[width=7.5in]{field_counts.pdf}
\end{center}
\end{figure}
%\pgfplotstabletypeset[columns={[index]0,[index]1}, precision=4]{data.txt}
%Referenecs
%Appendix
\end{document}  
